\documentclass{article}
\usepackage{amsmath}
\begin{document}
\thispagestyle{empty}
\pagestyle{empty}  
% convert -density 150 -trim formulas.pdf formulas.png

Mean squared displacement is the usual definition:
%
$$\mbox{MSD}(\tau)= \langle | \mathbf r (\tau) - \mathbf r(0) |^2 \rangle $$

The average, written explicitly, is done over windows of span $\tau$
that can be constructed in the interval $[w_f \dots w_t]$. To reduce
computation requirements, this average will be done with a stride of $w_s$ (i.e.\
only one window every $w_s$ will be considered):
%
$$\mbox{MSD}(\tau)= \underbrace{ 
  \frac{w_s}{w_t-w_f-\tau} 
  \sum_{t_0=w_f \mbox{ \scriptsize every } w_s}^{w_t-\tau-1}
 }_{\mbox{Average over intervals of span $\tau$ }}  
 \underbrace{ 
   \frac{1}{N}
   \sum_{i=1}^N
   \left(
     \mathbf r_i(t_0+\tau) - \mathbf r_i(t_0) 
   \right)^2
 }_{\mbox{Average over selected atoms}}
$$

When the center of mass drift subtraction is enabled, the following
replacement is applied to the above formula:
%
$$\mathbf r(t) \to \mathbf r(t)-\mathbf r_0(t)$$
%
where $\mathbf r_0(t)$ is the position of the center
of mass of the selected atoms at time $t$:
%
$$\mathbf r_0(t)=\sum_{i=1}^N \mathbf r_i(t) /N$$




\end{document}

